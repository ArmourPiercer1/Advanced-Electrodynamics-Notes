\documentclass[12pt,a4paper]{article}

% 中文字体支持
\usepackage{ctex}
\usepackage{xeCJK}

% 数学公式支持
\usepackage{amsmath,amssymb,amsthm}
\usepackage{mathtools}
\usepackage{bm} % 粗体数学符号

% 图表支持
\usepackage{graphicx}
\usepackage{tikz}
\usepackage{pgfplots}
\pgfplotsset{compat=1.17}

% 页面布局
\usepackage{geometry}
\geometry{left=2.5cm,right=2.5cm,top=3cm,bottom=3cm}
\usepackage{fancyhdr}
\usepackage{lastpage}

% 目录和超链接
\usepackage{hyperref}
\hypersetup{
    colorlinks=true,
    linkcolor=blue,
    filecolor=magenta,      
    urlcolor=cyan,
    citecolor=red
}

% 代码支持
\usepackage{listings}
\usepackage{xcolor}

% 表格支持
\usepackage{booktabs}
\usepackage{multirow}
\usepackage{array}

% 定理环境
\newtheorem{theorem}{定理}[section]
\newtheorem{lemma}[theorem]{引理}
\newtheorem{corollary}[theorem]{推论}
\newtheorem{proposition}[theorem]{命题}
\newtheorem{definition}[theorem]{定义}
\newtheorem{example}[theorem]{例题}
\newtheorem{exercise}[theorem]{练习}

% 页眉页脚设置
\setlength{\headheight}{14.5pt}
\pagestyle{fancy}
\fancyhf{}
\fancyhead[L]{高等电动力学课程总结}
\fancyhead[R]{\today}
\fancyfoot[C]{\thepage/\pageref{LastPage}}

% 代码样式设置
\lstset{
    basicstyle=\ttfamily\small,
    keywordstyle=\color{blue}\bfseries,
    commentstyle=\color{green!60!black},
    stringstyle=\color{red},
    showstringspaces=false,
    numbers=left,
    numberstyle=\tiny\color{gray},
    frame=single,
    breaklines=true
}

% 文档信息
\title{高等电动力学课程总结笔记}
\author{学生姓名}
\date{\today}
\begin{document}

\maketitle
\tableofcontents
\newpage

% ===== 课程总结主体内容 =====

\section*{课程概述}

\subsection*{课程目标}
高等电动力学是物理和电子工程专业的重要基础课程,本课程旨在:
\begin{itemize}
    \item 深入理解电磁场的基本理论和数学描述
    \item 掌握电磁波的传播、反射和折射规律
    \item 学习各种导波系统的特性分析
    \item 了解运动电荷的场和辐射现象
\end{itemize}

\subsection*{课程结构}
本课程共包含13个主要章节,从基础理论到实际应用,系统涵盖了高等电动力学的核心内容。

\section*{数学基础知识}
\subsection{重要物理常数}
\begin{itemize}
    \item 真空介电常数:$\varepsilon_0 = 8.854 \times 10^{-12}$ F/m
    \item 真空磁导率:$\mu_0 = 4\pi \times 10^{-7}$ H/m
    \item 光速:$c = \frac{1}{\sqrt{\mu_0 \varepsilon_0}} = 3 \times 10^8$ m/s
    \item 电子电荷:$e = 1.602 \times 10^{-19}$ C
    \item 普朗克常数:$h = 6.626 \times 10^{-34}$ J·s
\end{itemize}

\subsection{矢量分析:参见Jackson书扉页}
\subsubsection{矢量运算律}
混合积:
\begin{align}
    \mathbf{a\cdot(b \times c)=b \cdot (c \times a)=c \cdot(a \times b)} \\
\end{align}

三矢量的矢积:
\begin{align}
\mathbf{(a\times b)\times c=b(c\cdot a)-a(c\cdot b)} \\
\mathbf{a\times(b\times c)=b(c\cdot a)-c(a\cdot b)}
\end{align}

对称点积-叉积混合:
\begin{align}
    \mathbf{(a\times b)\cdot(c\times d)=(a\cdot c)(b\cdot d)-(a\cdot d)(b\cdot c)} \\
\end{align}

\subsection{nabla算符}
\subsubsection{各种坐标系下的nabla算符展开}

\subsection{nabla算符与矢量积的运算律}
\begin{align}
    \nabla\frac{1}{r}=-\frac{\mathbf{r}}{r^3}\\
    \nabla^2\left(\frac{1}{r}\right) = -4\pi \delta(\mathbf{r})
\end{align}

\subsection{偏微分方程}

\subsection{特殊函数}
\subsubsection{delta函数}
\begin{align}
    \int f(x)\delta'(x-a)= -f'(a)\\
    \delta (f(x))=\sum_i\frac{1}{\left|f'(x_i)\right|}\delta(x-x_i)
\end{align}
\subsubsection{Bessel函数}
\subsubsection{Legendre多项式}

\subsection{张量代数的基本规则}


\section{第一章:基础知识回顾}

\subsection{麦克斯韦方程组}
\begin{equation}
    \nabla \cdot \mathbf{E} = \frac{\rho}{\varepsilon_0}
\end{equation}
\begin{equation}
    \nabla \cdot \mathbf{B} = 0
\end{equation}
\begin{equation}
    \nabla \times \mathbf{E} = -\frac{\partial \mathbf{B}}{\partial t}
\end{equation}
\begin{equation}
    \nabla \times \mathbf{B} = \mu_0 \mathbf{J} + \mu_0 \varepsilon_0 \frac{\partial \mathbf{E}}{\partial t}
\end{equation}

\subsection{边界条件}
\begin{itemize}
    \item 电场的切向分量连续:$\mathbf{E}_{1t} = \mathbf{E}_{2t}$
    \item 磁场的法向分量连续:$\mathbf{B}_{1n} = \mathbf{B}_{2n}$
    \item 电位移的法向分量:$\mathbf{D}_{2n} - \mathbf{D}_{1n} = \sigma_f$
    \item 磁场强度的切向分量:$\mathbf{H}_{2t} - \mathbf{H}_{1t} = \mathbf{K}_f$
\end{itemize}

\section{静电场}
电场与电势:
\begin{align}
    \mathbf{E} = -\nabla \phi
\end{align}


\subsection{静电场的基本性质}
\subsubsection{叠加原理}
\begin{align}
    \mathbf{E}(\mathbf{r}) =& \frac{1}{4\pi\varepsilon_0} \int \frac{\rho(\mathbf{r}')(\mathbf{r}-\mathbf{r}')}{|\mathbf{r}-\mathbf{r}'|^3} dV'\\
    \phi(\mathbf{r}) =& \frac{1}{4\pi\varepsilon_0} \int \frac{\rho(\mathbf{r}')}{|\mathbf{r}-\mathbf{r}'|} dV'
\end{align}


\subsubsection{泊松方程和拉普拉斯方程}
\begin{equation}
    \nabla^2 \phi = -\frac{\rho}{\varepsilon}
\end{equation}
\subsection{高斯定理}
积分形式:
\begin{align}
    \oint_{S} \mathbf{E} \cdot d\mathbf{A} = \frac{Q_{\text{enc}}}{\varepsilon_0}
\end{align}
微分形式:
\begin{align}
    \nabla \cdot \mathbf{E} = \frac{\rho}{\varepsilon_0}
\end{align}

\subsubsection{静电场的边界条件}
1、面电荷两侧:电场和电位移矢量的切向分量连续,法向分量有跳变:
\begin{align}
    \mathbf{E}_{1t} = \mathbf{E}_{2t} \\
    \mathbf{D}_{2n} - \mathbf{D}_{1n} = \sigma_f
\end{align}

2、偶电层两侧:电势有跳变
\begin{align}
    \phi_2 - \phi_1 = \frac{D}{\varepsilon_0}
\end{align}

\subsubsection{静电场的能量}
点电荷场:
\begin{align}
    W = \frac{1}{2} \sum_{i} q_i \phi(\mathbf{r}_i)
\end{align}

体电荷场:
\begin{align}
    W=&\frac{1}{8\pi\varepsilon_0} \int \int \frac{\rho(\mathbf{r}) \rho(\mathbf{r}')}{|\mathbf{r}-\mathbf{r}'|} dV dV'\\
    =&\frac{1}{2} \int \rho(\mathbf{r}) \phi(\mathbf{r}) dV\\
    =&-\frac{\varepsilon_0}{2}\int \phi\nabla^2\phi dV\\
    =&\frac{\varepsilon_0}{2}\int |\nabla \phi|^2 dV\\
    =&\frac{\varepsilon_0}{2}\int |\mathbf{E}|^2 dV
\end{align}

能量密度
\begin{align}
    w=\frac{1}{2} \varepsilon_0 |\mathbf{E}|^2
\end{align}


\subsubsection{导体表面和介质界面的静电场}
对于导体表面的静电荷场:
\begin{align}
    \sigma=-\varepsilon_0\frac{\partial \Phi}{\partial n}
\end{align}
式中,$\Phi$为真空区的电势。

导体表面受到的静电压强:
\begin{align}
    \mathbf{P}=\frac{1}{2} \sigma E_n \mathbf{n}=\frac{1}{2} \varepsilon_0 E_n^2 \mathbf{n}
\end{align}


\subsection{静电场的求解方法}
各种不同方法的基础:唯一性定理
\begin{theorem}[唯一性定理]
    \item 假设区域V内给定电荷分布,区域V的边界S上给定电势或者$\partial \phi/\partial n$,则区域V内的电势$\phi$唯一确定。
    \item 假设区域V内有一些导体,导体之外的电荷分布已知,导体上的总电荷已知,区域的边界上电势$\Phi$或者$\partial \phi/\partial n$已知,则区域V内的电势$\phi$唯一确定。
\end{theorem}



\subsubsection{分离变量法}
在直角坐标系、柱坐标系和球坐标系中的应用。

\subsubsection{镜像法}
\begin{itemize}
    \item 导体平面的镜像法
    \item 导体球面的镜像法
    \item 柱状导体的镜像法
\end{itemize}

\subsubsection{格林函数法(需要进一步加深理解)}

\begin{theorem}[格林恒等式]
    \begin{align}
    \int_V \left(\phi \nabla^2 \psi + \nabla \phi \cdot \nabla \psi\right) dV = \oint_S \phi \frac{\partial \psi}{\partial n} dS
    \end{align}
    
    \begin{align}
    \int_V \left(\phi \nabla^2 \psi - \psi \nabla^2 \phi\right) dV = \oint_S \left(\phi \frac{\partial \psi}{\partial n} - \psi \frac{\partial \phi}{\partial n}\right) dS
    \end{align}
\end{theorem}

对任意确定的边值问题,我们想找到的格林函数总是满足如下方程:
\begin{equation}
    \nabla^2 G(\mathbf{r}, \mathbf{r}') = -4\pi \delta(\mathbf{r} - \mathbf{r}')
\end{equation}
边界条件:
\begin{itemize}
    \item 第一类边界条件(Dirichlet边界条件):$G(\mathbf{r}, \mathbf{r}') = 0$,当$\mathbf{r}'$在边界$S$上时
    \item 第二类边界条件(Neumann边界条件):$\frac{\partial G(\mathbf{r}, \mathbf{r}')}{\partial n'} = -\frac{4\pi}{S}$,当$\mathbf{r}'$在边界$S$上时
\end{itemize}

则对于任意给定的格林函数$G (\mathbf{r}, \mathbf{r}')$,静电势的解为:
\begin{align*}
    \phi(\mathbf{r}) =& \frac{1}{4\pi\varepsilon_0} \int_V \rho(\mathbf{r}') G(\mathbf{r}, \mathbf{r}') dV' 
    + \frac{1}{4\pi} \oint_S \left( G(\mathbf{r}, \mathbf{r}') 
    \frac{\partial \phi}{\partial n'} - \phi(\mathbf{r}') 
    \frac{\partial G(\mathbf{r}, \mathbf{r}')}{\partial n'} \right) dS'\\ \\
    =&\frac{1}{4\pi\varepsilon_0} \int_V \rho(\mathbf{r}') G_D(\mathbf{r}, \mathbf{r}') dV'
    + \frac{1}{4\pi} \oint_S \phi(\mathbf{r}') \frac{\partial G_D(\mathbf{r}, \mathbf{r}')}{\partial n'} dS' \quad \text{(Dirichlet边界条件)}\\ \\
    =&\langle\phi\rangle_S+\frac{1}{4\pi\varepsilon_0} \int_V \rho(\mathbf{r}') G_N(\mathbf{r}, \mathbf{r}') dV'
     + \frac{1}{4\pi} \oint_S G_N(\mathbf{r}, \mathbf{r}') \frac{\partial \phi}{\partial n'} dS' \quad \text{(Neumann边界条件)}
\end{align*}


\subsubsection{多极展开}
电势的多极展开表达式:
\begin{equation}
    \phi(\mathbf{r}) = \frac{1}{4\pi\varepsilon_0} \sum_{l=0}^{\infty} \frac{1}{r^{l+1}} \sum_{m=-l}^{l} Q_{lm} Y_{lm}(\theta, \phi)
\end{equation}

\subsection{电偶极子、电四极子}
高阶多极矩的计算和物理意义。



\section{静磁场}
\subsection{静磁场的基本性质}
\subsubsection{比奥-萨伐尔定律}
\begin{equation}
    \mathbf{B}(\mathbf{r}) = \frac{\mu_0}{4\pi} \int \frac{\mathbf{J}(\mathbf{r}') \times (\mathbf{r} - \mathbf{r}')}{|\mathbf{r} - \mathbf{r}'|^3} dV'
\end{equation}
\subsubsection{安培环路定理}
积分形式:
\begin{align}
    \oint_{C} \mathbf{B} \cdot d\mathbf{l} = \mu_0 I_{\text{enc}}
\end{align}
微分形式:
\begin{align}
    \nabla \times \mathbf{B} = \mu_0 \mathbf{J}
\end{align}
\subsubsection{磁矢势}
\begin{equation}
    \mathbf{B} = \nabla \times \mathbf{A}
\end{equation}
\begin{equation}
    \mathbf{A}(\mathbf{r}) = \frac{\mu_0}{4\pi} \int \frac{\mathbf{J}(\mathbf{r}')}{|\mathbf{r} - \mathbf{r}'|} dV'
\end{equation}
\subsection{磁介质中的静磁场}
\subsubsection{磁化强度和磁极化强度}
\begin{align}
    \mathbf{M} = \frac{\text{磁矩}}{\text{体积}}
\end{align}
\subsubsection{磁场强度}
\begin{align}
    \mathbf{H} = \frac{1}{\mu_0} \mathbf{B} - \mathbf{M}
\end{align}
\subsubsection{磁介质中的边界条件}
\begin{align}
    \mathbf{B}_{1n} = \mathbf{B}_{2n} \\
    \mathbf{H}_{2t} - \mathbf{H}_{1t} = \mathbf{K}_f
\end{align}
\subsection{静磁场的能量}
\begin{align}
    W = \frac{1}{2} \int \mathbf{J} \cdot \mathbf{A} dV
\end{align}
\begin{align}
    W = \frac{1}{2\mu_0} \int |\mathbf{B}|^2 dV
\end{align}


\section{电磁场的基本性质}
\subsection{能量与能流}
\begin{itemize}
    \item 电磁场的能量密度:$u = \frac{1}{2}(\varepsilon E^2 + \frac{1}{\mu} B^2)$
    \item 坡印廷矢量:$\mathbf{S} = \mathbf{E} \times \mathbf{H}$
    \item 能量守恒方程:$\frac{\partial u}{\partial t} + \nabla \cdot \mathbf{S} = -\mathbf{J} \cdot \mathbf{E}$
\end{itemize}

\subsection{动量}
\begin{itemize}
    \item 电磁场的动量密度:$\mathbf{g} = \varepsilon \mathbf{E} \times \mathbf{B}$
    \item 电磁力密度:$\mathbf{f} = \rho \mathbf{E} + \mathbf{J} \times \mathbf{B}$
    \item 动量守恒方程:$\frac{\partial \mathbf{g}}{\partial t} + \nabla \cdot \mathbf{T} = -\mathbf{f}$
\end{itemize}

\subsection{角动量}
\begin{itemize}
    \item 电磁场的角动量密度:$\mathbf{l} = \mathbf{r} \times \mathbf{g}$
    \item 角动量守恒方程:$\frac{\partial \mathbf{l}}{\partial t} + \nabla \cdot \mathbf{M} = -\mathbf{r} \times \mathbf{f}$
    \item 角动量密度张量:$\mathbf{M} = \mathbf{r} \times \mathbf{T}$
\end{itemize}


\section{电磁波}

\subsection{波动方程}
从麦克斯韦方程组导出的波动方程:
\begin{equation}
    \nabla^2 \mathbf{E} - \mu\varepsilon \frac{\partial^2 \mathbf{E}}{\partial t^2} = 0
\end{equation}

\subsection{平面电磁波}
\begin{itemize}
    \item 均匀平面波的传播特性
    \item 波阻抗和能流密度
    \item 偏振态的描述
\end{itemize}

\subsection{电磁波的能量和动量}
\begin{itemize}
    \item 能量密度:$u = \frac{1}{2}(\varepsilon E^2 + \mu H^2)$
    \item 坡印廷矢量:$\mathbf{S} = \mathbf{E} \times \mathbf{H}$
    \item 动量密度:$\mathbf{g} = \varepsilon \mathbf{E} \times \mathbf{B}$
\end{itemize}

\section{电磁波的反射和折射}

\subsection{电磁波的边界条件}
\subsubsection{理想金属边界条件}
\subsubsection{有损耗金属边界条件}
\subsubsection{介质交界面条件}

\subsection{斯涅尔定律}
\begin{equation}
    n_1 \sin\theta_1 = n_2 \sin\theta_2
\end{equation}

\subsection{菲涅尔公式}
\begin{itemize}
    \item TE波(s偏振)的反射和透射系数
    \item TM波(p偏振)的反射和透射系数
    \item 布儒斯特角和全反射现象
\end{itemize}

\subsection{导波系统简介}
传输线、波导管等导波系统的基本概念。

\section{波导基本理论}
\subsection{从麦克斯韦方程组导出的波动方程}
\subsubsection{直角坐标系的波动方程及其基本解}
\subsubsection{柱坐标系的波动方程及其基本解}

\subsection{纵向均匀系统的纵横关系}

\subsection{波导的阻抗与功率传输}

\section{几种常见波导}

\subsection{矩形波导}


\subsubsection{矩形波导的模式}
\subsubsection{截止频率和传播特性}
\subsubsection{TE模式场分布}
\subsubsection{TM模式场分布}


% \subsubsection{波导的传播常数}
% \begin{equation}
%     \beta = \sqrt{k^2 - k_c^2}
% \end{equation}

\subsubsection{波导的阻抗和功率传输}

\subsection{圆波导}


\subsubsection{圆波导的模式}
\subsubsection{截止频率和传播特性}
\subsubsection{TE模式场分布}
\subsubsection{TM模式场分布}


% \subsubsection{波导的传播常数}
% \begin{equation}
%     \beta = \sqrt{k^2 - k_c^2}
% \end{equation}

\subsubsection{波导的阻抗和功率传输}
\subsubsection{高阶模式的特性}

\subsection{介质加载波导}

\subsubsection{部分填充介质的波导}
介质对传播常数的影响。

\subsubsection{特征方程的求解}
不同边界条件下的特征方程。

\subsection{介质加载圆波导和介质波导}

\subsubsection{介质圆波导}
\begin{itemize}
    \item 介质波导的导模条件
    \item 泄漏模和辐射模
\end{itemize}

\subsection{光纤原理}
光在光纤中的传播机制。

\section{谐振腔基础理论}

\subsection{波动方程及其基本解}

\subsection{谐振系统的纵横关系}

\subsubsection{波导的储能与品质因数}

\section{常见谐振腔}
\subsection{矩形谐振腔}
\begin{itemize}
    \item 谐振频率的计算
    \item 品质因数Q的定义和计算
    \item 模式的正交性
\end{itemize}

\subsection{圆柱谐振腔}
圆柱形谐振腔的谐振特性。

\section{第十一章:谐振腔链和空间谐波}

\subsection{耦合谐振腔}
多个谐振腔的耦合效应。

\subsection{周期结构}
\begin{itemize}
    \item 布洛赫定理
    \item 色散关系
    \item 带隙结构
\end{itemize}

\section{矢势和运动电荷的场}

\subsection{电磁势}
\begin{itemize}
    \item 标势$\phi$和矢势$\mathbf{A}$的定义
    \item 洛伦兹规范条件:$\nabla \cdot \mathbf{A} + \frac{1}{c^2}\frac{\partial \phi}{\partial t} = 0$
\end{itemize}

\subsection{达朗贝尔方程}
\begin{equation}
    \nabla^2 \phi - \frac{1}{c^2}\frac{\partial^2 \phi}{\partial t^2} = -\frac{\rho}{\varepsilon_0}
\end{equation}

\subsection{李纳-维谢尔势}
运动电荷产生的电磁势。

\section{运动电荷的场和同步辐射}

\subsection{加速电荷的辐射}
\begin{itemize}
    \item 拉莫尔公式
    \item 相对论性推广
\end{itemize}

\subsection{同步辐射}
\begin{itemize}
    \item 同步辐射的特性
    \item 角分布和频谱
    \item 应用领域
\end{itemize}

\section{重要公式汇总}

\subsection{基本常数}
\begin{itemize}
    \item 真空介电常数:$\varepsilon_0 = 8.854 \times 10^{-12}$ F/m
    \item 真空磁导率:$\mu_0 = 4\pi \times 10^{-7}$ H/m
    \item 光速:$c = \frac{1}{\sqrt{\mu_0 \varepsilon_0}} = 3 \times 10^8$ m/s
\end{itemize}

\subsection{常用矢量恒等式}
\begin{itemize}
    \item $\nabla \cdot (\nabla \times \mathbf{A}) = 0$
    \item $\nabla \times (\nabla \phi) = 0$
    \item $\nabla \times (\nabla \times \mathbf{A}) = \nabla(\nabla \cdot \mathbf{A}) - \nabla^2 \mathbf{A}$
\end{itemize}

\section{学习建议和复习要点}

\subsection{重点掌握}
\begin{enumerate}
    \item 麦克斯韦方程组及其物理意义
    \item 电磁波的传播特性和边界条件
    \item 各种导波系统的模式分析
    \item 运动电荷的辐射特性
\end{enumerate}

\subsection{学习方法}
\begin{itemize}
    \item 理论推导与物理图像相结合
    \item 多做习题加深理解
    \item 注意不同坐标系下的数学技巧
    \item 建立知识框架,融会贯通
\end{itemize}

% ===== 参考文献部分 =====
\begin{thebibliography}{99}
\bibitem{jackson} J.D. Jackson, \emph{Classical Electrodynamics}, 3rd ed., Wiley, 1998.
\bibitem{griffiths} D.J. Griffiths, \emph{Introduction to Electrodynamics}, 4th ed., Cambridge University Press, 2017.
\bibitem{guo} 郭硕鸿, \emph{电动力学}, 高等教育出版社, 2008.
\bibitem{born} M. Born and E. Wolf, \emph{Principles of Optics}, 7th ed., Cambridge University Press, 1999.
\end{thebibliography}

\end{document}
