\documentclass{article}

% 中文字体支持
\usepackage{ctex}
\usepackage{xeCJK}

% 数学公式支持
\usepackage{amsmath,amssymb,amsthm}
\usepackage{mathtools}
\usepackage{bm} % 粗体数学符号

% 图表支持
\usepackage{graphicx}
\usepackage{tikz}
\usepackage{pgfplots}
\pgfplotsset{compat=1.17}
\usepackage{float}

% 页面布局
\usepackage{geometry}
\geometry{left=2.5cm,right=2.5cm,top=3cm,bottom=3cm}
\usepackage{fancyhdr}
\usepackage{lastpage}

% 目录和超链接
\usepackage{hyperref}
\hypersetup{
    colorlinks=true,
    linkcolor=blue,
    filecolor=magenta,      
    urlcolor=cyan,
    citecolor=red
}

% 代码支持
\usepackage{listings}
\usepackage{xcolor}

% 表格支持
\usepackage{booktabs}
\usepackage{multirow}
\usepackage{array}

% 定理环境
\newtheorem{theorem}{定理}[section]
\newtheorem{lemma}[theorem]{引理}
\newtheorem{corollary}[theorem]{推论}
\newtheorem{proposition}[theorem]{命题}
\newtheorem{definition}[theorem]{定义}
\newtheorem{example}[theorem]{例题}
\newtheorem{exercise}[theorem]{练习}

% 页眉页脚设置
\setlength{\headheight}{14.5pt}
\pagestyle{fancy}
\fancyhf{}
\fancyhead[L]{高等电动力学课程总结}
\fancyhead[R]{\today}
\fancyfoot[C]{\thepage/\pageref{LastPage}}

% 代码样式设置
\lstset{
    basicstyle=\ttfamily\small,
    keywordstyle=\color{blue}\bfseries,
    commentstyle=\color{green!60!black},
    stringstyle=\color{red},
    showstringspaces=false,
    numbers=left,
    numberstyle=\tiny\color{gray},
    frame=single,
    breaklines=true
}

% 文档信息
\title{高等电动力学课程总结笔记}
\author{学生姓名}
\date{\today}


\begin{document}

\maketitle
\tableofcontents
\newpage

% ===== 课程总结主体内容 =====

\section*{课程概述}

\subsection*{课程目标}
高等电动力学是物理和电子工程专业的重要基础课程,本课程旨在:
\begin{itemize}
    \item 深入理解电磁场的基本理论和数学描述
    \item 掌握电磁波的传播、反射和折射规律
    \item 学习各种导波系统的特性分析
    \item 了解运动电荷的场和辐射现象
\end{itemize}

\subsection*{课程结构}
本课程共包含13个主要章节,从基础理论到实际应用,系统涵盖了高等电动力学的核心内容。

\section*{数学基础知识}
\subsection{重要常数}

物理常数
\begin{itemize}
    \item 真空介电常数:$\varepsilon_0 = 8.854 \times 10^{-12}$ F/m
    \item 真空磁导率:$\mu_0 = 4\pi \times 10^{-7}$ H/m
    \item 光速:$c = \frac{1}{\sqrt{\mu_0 \varepsilon_0}} = 3 \times 10^8$ m/s
    \item 电子电荷:$e = 1.602 \times 10^{-19}$ C
    \item 普朗克常数:$h = 6.626 \times 10^{-34}$ J·s
\end{itemize}

材料参数
\begin{itemize}
    \item 铜的电导率$\sigma_{Cu} = 5.8 \times 10^7$ S/m
\end{itemize}

\subsection{矢量分析:参见Jackson书扉页}
\subsubsection{矢量运算律}
混合积:
\begin{align}
    \mathbf{a\cdot(b \times c)=b \cdot (c \times a)=c \cdot(a \times b)} \\
\end{align}

三矢量的矢积:
\begin{align}
\mathbf{(a\times b)\times c=b(c\cdot a)-a(c\cdot b)} \\
\mathbf{a\times(b\times c)=b(c\cdot a)-c(a\cdot b)}
\end{align}

对称点积-叉积混合:
\begin{align}
    \mathbf{(a\times b)\cdot(c\times d)=(a\cdot c)(b\cdot d)-(a\cdot d)(b\cdot c)} \\
\end{align}

\subsection{nabla算符}
\subsubsection{各种坐标系下的nabla算符展开}

\subsection{nabla算符与矢量积的运算律}
\begin{align}
    \nabla\frac{1}{r}=-\frac{\mathbf{r}}{r^3}\\
    \nabla^2\left(\frac{1}{r}\right) = -4\pi \delta(\mathbf{r})
\end{align}

\subsection{偏微分方程}

\subsection{特殊函数}
\subsubsection{delta函数}
\begin{align}
    \int f(x)\delta'(x-a)= -f'(a)\\
    \delta (f(x))=\sum_i\frac{1}{\left|f'(x_i)\right|}\delta(x-x_i)
\end{align}
\subsubsection{Bessel函数}

\begin{itemize}
    \item 重要积分规律:若$a$为$J_n(k_cr)$或$J_n'(k_cr)$的零点,则有
    \begin{align}
        \int_0^a J_n^2(k_cr)rdr=\int_0^a J_n'^2(k_cr)rdr=\frac{a^2}{2}[J_n'^2(k_ca)+J_n^2(k_ca)]
    \end{align}
\end{itemize}



\subsubsection{Legendre多项式}

\subsection{张量代数的基本规则}

\subsection{材料性质参数计算}
\begin{itemize}
    \item 趋肤深度
    \begin{align}
        \delta = \sqrt{\frac{2}{\omega \mu \sigma}}
    \end{align}
    \item 表面电阻
    \begin{align}
        R_s=\frac{1}{\sigma \delta}=\sqrt{\frac{\omega \mu}{2 \sigma}}
    \end{align}
\end{itemize}


\section{第一章:基础知识回顾}

\subsection{麦克斯韦方程组}
\begin{equation}
    \nabla \cdot \mathbf{E} = \frac{\rho}{\varepsilon_0}
\end{equation}
\begin{equation}
    \nabla \cdot \mathbf{B} = 0
\end{equation}
\begin{equation}
    \nabla \times \mathbf{E} = -\frac{\partial \mathbf{B}}{\partial t}
\end{equation}
\begin{equation}
    \nabla \times \mathbf{B} = \mu_0 \mathbf{J} + \mu_0 \varepsilon_0 \frac{\partial \mathbf{E}}{\partial t}
\end{equation}

\subsection{边界条件}
\begin{itemize}
    \item 电场的切向分量连续:$\mathbf{E}_{1t} = \mathbf{E}_{2t}$
    \item 磁场的法向分量连续:$\mathbf{B}_{1n} = \mathbf{B}_{2n}$
    \item 电位移的法向分量:$\mathbf{D}_{2n} - \mathbf{D}_{1n} = \sigma_f$
    \item 磁场强度的切向分量:$\mathbf{H}_{2t} - \mathbf{H}_{1t} = \mathbf{K}_f$
\end{itemize}

\section{静电场}
电场与电势:
\begin{align}
    \mathbf{E} = -\nabla \phi
\end{align}


\subsection{静电场的基本性质}
\subsubsection{叠加原理}
\begin{align}
    \mathbf{E}(\mathbf{r}) =& \frac{1}{4\pi\varepsilon_0} \int \frac{\rho(\mathbf{r}')(\mathbf{r}-\mathbf{r}')}{|\mathbf{r}-\mathbf{r}'|^3} dV'\\
    \phi(\mathbf{r}) =& \frac{1}{4\pi\varepsilon_0} \int \frac{\rho(\mathbf{r}')}{|\mathbf{r}-\mathbf{r}'|} dV'
\end{align}


\subsubsection{泊松方程和拉普拉斯方程}
\begin{equation}
    \nabla^2 \phi = -\frac{\rho}{\varepsilon}
\end{equation}
\subsection{高斯定理}
积分形式:
\begin{align}
    \oint_{S} \mathbf{E} \cdot d\mathbf{A} = \frac{Q_{\text{enc}}}{\varepsilon_0}
\end{align}
微分形式:
\begin{align}
    \nabla \cdot \mathbf{E} = \frac{\rho}{\varepsilon_0}
\end{align}

\subsubsection{静电场的边界条件}
1、面电荷两侧:电场和电位移矢量的切向分量连续,法向分量有跳变:
\begin{align}
    \mathbf{E}_{1t} = \mathbf{E}_{2t} \\
    \mathbf{D}_{2n} - \mathbf{D}_{1n} = \sigma_f
\end{align}

2、偶电层两侧:电势有跳变
\begin{align}
    \phi_2 - \phi_1 = \frac{D}{\varepsilon_0}
\end{align}

\subsubsection{静电场的能量}
点电荷场:
\begin{align}
    W = \frac{1}{2} \sum_{i} q_i \phi(\mathbf{r}_i)
\end{align}

体电荷场:
\begin{align}
    W=&\frac{1}{8\pi\varepsilon_0} \int \int \frac{\rho(\mathbf{r}) \rho(\mathbf{r}')}{|\mathbf{r}-\mathbf{r}'|} dV dV'\\
    =&\frac{1}{2} \int \rho(\mathbf{r}) \phi(\mathbf{r}) dV\\
    =&-\frac{\varepsilon_0}{2}\int \phi\nabla^2\phi dV\\
    =&\frac{\varepsilon_0}{2}\int |\nabla \phi|^2 dV\\
    =&\frac{\varepsilon_0}{2}\int |\mathbf{E}|^2 dV
\end{align}

能量密度
\begin{align}
    w=\frac{1}{2} \varepsilon_0 |\mathbf{E}|^2
\end{align}


\subsubsection{导体表面和介质界面的静电场}
对于导体表面的静电荷场:
\begin{align}
    \sigma=-\varepsilon_0\frac{\partial \Phi}{\partial n}
\end{align}
式中,$\Phi$为真空区的电势。

导体表面受到的静电压强:
\begin{align}
    \mathbf{P}=\frac{1}{2} \sigma E_n \mathbf{n}=\frac{1}{2} \varepsilon_0 E_n^2 \mathbf{n}
\end{align}


\subsection{静电场的求解方法}
各种不同方法的基础:唯一性定理
\begin{theorem}[唯一性定理]
    \item 假设区域V内给定电荷分布,区域V的边界S上给定电势或者$\partial \phi/\partial n$,则区域V内的电势$\phi$唯一确定。
    \item 假设区域V内有一些导体,导体之外的电荷分布已知,导体上的总电荷已知,区域的边界上电势$\Phi$或者$\partial \phi/\partial n$已知,则区域V内的电势$\phi$唯一确定。
\end{theorem}



\subsubsection{分离变量法}
在直角坐标系、柱坐标系和球坐标系中的应用。

\subsubsection{镜像法}
\begin{itemize}
    \item 导体平面的镜像法
    \item 导体球面的镜像法
    \item 柱状导体的镜像法
\end{itemize}

\subsubsection{格林函数法(需要进一步加深理解)}

\begin{theorem}[格林恒等式]
    \begin{align}
    \int_V \left(\phi \nabla^2 \psi + \nabla \phi \cdot \nabla \psi\right) dV = \oint_S \phi \frac{\partial \psi}{\partial n} dS
    \end{align}
    
    \begin{align}
    \int_V \left(\phi \nabla^2 \psi - \psi \nabla^2 \phi\right) dV = \oint_S \left(\phi \frac{\partial \psi}{\partial n} - \psi \frac{\partial \phi}{\partial n}\right) dS
    \end{align}
\end{theorem}

对任意确定的边值问题,我们想找到的格林函数总是满足如下方程:
\begin{equation}
    \nabla^2 G(\mathbf{r}, \mathbf{r}') = -4\pi \delta(\mathbf{r} - \mathbf{r}')
\end{equation}
边界条件:
\begin{itemize}
    \item 第一类边界条件(Dirichlet边界条件):$G(\mathbf{r}, \mathbf{r}') = 0$,当$\mathbf{r}'$在边界$S$上时
    \item 第二类边界条件(Neumann边界条件):$\frac{\partial G(\mathbf{r}, \mathbf{r}')}{\partial n'} = -\frac{4\pi}{S}$,当$\mathbf{r}'$在边界$S$上时
\end{itemize}

则对于任意给定的格林函数$G (\mathbf{r}, \mathbf{r}')$,静电势的解为:
\begin{align*}
    \phi(\mathbf{r}) =& \frac{1}{4\pi\varepsilon_0} \int_V \rho(\mathbf{r}') G(\mathbf{r}, \mathbf{r}') dV' 
    + \frac{1}{4\pi} \oint_S \left( G(\mathbf{r}, \mathbf{r}') 
    \frac{\partial \phi}{\partial n'} - \phi(\mathbf{r}') 
    \frac{\partial G(\mathbf{r}, \mathbf{r}')}{\partial n'} \right) dS'\\ \\
    =&\frac{1}{4\pi\varepsilon_0} \int_V \rho(\mathbf{r}') G_D(\mathbf{r}, \mathbf{r}') dV'
    + \frac{1}{4\pi} \oint_S \phi(\mathbf{r}') \frac{\partial G_D(\mathbf{r}, \mathbf{r}')}{\partial n'} dS' \quad \text{(Dirichlet边界条件)}\\ \\
    =&\langle\phi\rangle_S+\frac{1}{4\pi\varepsilon_0} \int_V \rho(\mathbf{r}') G_N(\mathbf{r}, \mathbf{r}') dV'
     + \frac{1}{4\pi} \oint_S G_N(\mathbf{r}, \mathbf{r}') \frac{\partial \phi}{\partial n'} dS' \quad \text{(Neumann边界条件)}
\end{align*}


\subsubsection{多极展开}
电势的多极展开表达式:
\begin{equation}
    \phi(\mathbf{r}) = \frac{1}{4\pi\varepsilon_0} \sum_{l=0}^{\infty} \frac{1}{r^{l+1}} \sum_{m=-l}^{l} Q_{lm} Y_{lm}(\theta, \phi)
\end{equation}

\subsection{电偶极子、电四极子}
高阶多极矩的计算和物理意义。



\section{静磁场}
\subsection{静磁场的基本性质}
\subsubsection{比奥-萨伐尔定律}
\begin{equation}
    \mathbf{B}(\mathbf{r}) = \frac{\mu_0}{4\pi} \int \frac{\mathbf{J}(\mathbf{r}') \times (\mathbf{r} - \mathbf{r}')}{|\mathbf{r} - \mathbf{r}'|^3} dV'
\end{equation}
\subsubsection{安培环路定理}
积分形式:
\begin{align}
    \oint_{C} \mathbf{B} \cdot d\mathbf{l} = \mu_0 I_{\text{enc}}
\end{align}
微分形式:
\begin{align}
    \nabla \times \mathbf{B} = \mu_0 \mathbf{J}
\end{align}
\subsubsection{磁矢势}
\begin{equation}
    \mathbf{B} = \nabla \times \mathbf{A}
\end{equation}
\begin{equation}
    \mathbf{A}(\mathbf{r}) = \frac{\mu_0}{4\pi} \int \frac{\mathbf{J}(\mathbf{r}')}{|\mathbf{r} - \mathbf{r}'|} dV'
\end{equation}
\subsection{磁介质中的静磁场}
\subsubsection{磁化强度和磁极化强度}
\begin{align}
    \mathbf{M} = \frac{\text{磁矩}}{\text{体积}}
\end{align}
\subsubsection{磁场强度}
\begin{align}
    \mathbf{H} = \frac{1}{\mu_0} \mathbf{B} - \mathbf{M}
\end{align}
\subsubsection{磁介质中的边界条件}
\begin{align}
    \mathbf{B}_{1n} = \mathbf{B}_{2n} \\
    \mathbf{H}_{2t} - \mathbf{H}_{1t} = \mathbf{K}_f
\end{align}
\subsection{静磁场的能量}
\begin{align}
    W = \frac{1}{2} \int \mathbf{J} \cdot \mathbf{A} dV
\end{align}
\begin{align}
    W = \frac{1}{2\mu_0} \int |\mathbf{B}|^2 dV
\end{align}


\section{电磁场的基本性质}
\subsection{能量与能流}
\begin{itemize}
    \item 电磁场的能量密度:$u = \frac{1}{2}(\varepsilon E^2 + \frac{1}{\mu} B^2)$
    \item 坡印廷矢量:$\mathbf{S} = \mathbf{E} \times \mathbf{H}$
    \item 能量守恒方程:$\frac{\partial u}{\partial t} + \nabla \cdot \mathbf{S} = -\mathbf{J} \cdot \mathbf{E}$
\end{itemize}

\subsection{动量}
\begin{itemize}
    \item 电磁场的动量密度:$\mathbf{g} = \varepsilon \mathbf{E} \times \mathbf{B}$
    \item 电磁力密度:$\mathbf{f} = \rho \mathbf{E} + \mathbf{J} \times \mathbf{B}$
    \item 动量守恒方程:$\frac{\partial \mathbf{g}}{\partial t} + \nabla \cdot \mathbf{T} = -\mathbf{f}$
\end{itemize}

\subsection{角动量}
\begin{itemize}
    \item 电磁场的角动量密度:$\mathbf{l} = \mathbf{r} \times \mathbf{g}$
    \item 角动量守恒方程:$\frac{\partial \mathbf{l}}{\partial t} + \nabla \cdot \mathbf{M} = -\mathbf{r} \times \mathbf{f}$
    \item 角动量密度张量:$\mathbf{M} = \mathbf{r} \times \mathbf{T}$
\end{itemize}


\section{电磁波}

\subsection{波动方程}
从麦克斯韦方程组导出的波动方程:
\begin{equation}
    \nabla^2 \mathbf{E} - \mu\varepsilon \frac{\partial^2 \mathbf{E}}{\partial t^2} = 0
\end{equation}


对于单色波,导出Helmholtz方程:
\begin{align}
    \nabla^2 \mathbf{E} + k^2 \mathbf{E} = 0 \\
    \nabla^2 \mathbf{H} + k^2 \mathbf{H} = 0
\end{align}
式中,$k = \omega \sqrt{\mu\varepsilon}$为波数。


\subsection{平面电磁波}
最简单情况:均匀平面波电场矢量沿x方向,波沿z方向传播

方程组:
\begin{align}
    \frac{\partial^2 E_x}{\partial z^2} + \omega^2 \mu \varepsilon E_x = 0 \\
    E_x=E_+e^{ikz}+E_-e^{-ikz}
\end{align}
磁场 
\begin{align}
    \mathbf{H}=\frac{1}{\eta}Ee^{ikz}\hat{y}
\end{align}
式中,$\eta = \sqrt{\frac{\mu}{\varepsilon}}$为波阻抗。

\textbf{重要概念}:
\begin{itemize}
    \item 波阵面:空间中相位相同的点构成的曲面,即等相位面
    \item 平面波:等相位面为无限大平面的电磁波
    \item 相速度:电磁波的等相位面在空间中的移动速度,$v_p=\omega/k$
    \item 群速度:电磁波中能量和信息传递的速度, $v_g=\mathrm{d}\omega/\mathrm{d}k$
\end{itemize}

\subsubsection{电磁场关系}
均匀平面波的电磁场关系:
    
    \begin{align}
        \mathbf{E} =& \eta\cdot\mathbf{H}\times \hat{\mathbf{n}} \\
        \mathbf{H} =& \frac{1}{\eta}\cdot\hat{\mathbf{n}}\times \mathbf{E}
    \end{align}
    即:$\mathbf{E,H,\hat{n}}$三者互相垂直,构成右手系。

\subsubsection{波阻抗}
\begin{align}
        \eta = \frac{E}{H} = \sqrt{\frac{\mu}{\varepsilon}}
    \end{align}
    真空波阻抗
    \begin{align}
        \eta_0 = \sqrt{\frac{\mu_0}{\varepsilon_0}} =120\pi\Omega \approx 377 \Omega
    \end{align}

\subsubsection{能量密度与能流密度}
能量密度:理想媒质中均匀平面波的电场能量等于磁场能量
\begin{itemize}
    \item 电场能量密度:$u_E = \frac{1}{2} \varepsilon E^2$
    \item 磁场能量密度:$u_H = \frac{1}{2} \mu H^2=\frac{1}{2}\varepsilon E^2$
    \item 总能量密度:$u = u_E + u_H = \varepsilon E^2$
\end{itemize}

能流密度:
\begin{align}
    \mathbf{S} = \mathbf{E} \times \mathbf{H}=\frac{1}{\eta}|\mathbf{E}|^2\mathbf{n}
\end{align}

电磁波的能量传播速度:
\begin{align}
    v=\frac{|\mathbf{S}|}{u}   = \frac{1}{\sqrt{\mu\varepsilon}} 
\end{align}

\subsubsection{复数形式表达}
复数表示的场:
\begin{align}
    \mathbf{E}(\mathbf{r},t) =& \mathrm{Re}\{\dot{\mathbf{E}}(\mathbf{r})e^{-i\omega t}\} \\
    \mathbf{H}(\mathbf{r},t) =& \mathrm{Re}\{\dot{\mathbf{H}}(\mathbf{r})e^{-i\omega t}\}
\end{align}

复坡印廷矢量:
\begin{align}
    \dot{\mathbf{S}} = \frac{1}{2} \dot{\mathbf{E}} \times \dot{\mathbf{H}}^*
\end{align}

复数表示和实际物理量关系:
\begin{align}
    \mathbf{E}(\mathbf{r},t) =& \mathrm{Re}\{\dot{\mathbf{E}}(\mathbf{r})e^{-i\omega t}\} \\
    \mathbf{H}(\mathbf{r},t) =& \mathrm{Re}\{\dot{\mathbf{H}}(\mathbf{r})e^{-i\omega t}\}\\
    \mathbf{S}(\mathbf{r},t) =& \frac{1}{2}\mathrm{Re}\{\dot{\mathbf{E}} \times \dot{\mathbf{H}}^*
    +  \dot{\mathbf{E}}\times \dot{\mathbf{H}}e^{-i2\omega t}\}\\
    \langle \mathbf{S} \rangle =& \frac{1}{2}\mathrm{Re}\{\dot{\mathbf{S}}\}
\end{align}

即:复坡印廷矢量的实部为平均功率流密度,虚部为磁场和电场之间能量转换的无功功率流密度。

\subsection{电磁波的能量和动量}
\begin{itemize}
    \item 能量密度:$u = \frac{1}{2}(\varepsilon E^2 + \mu H^2)$
    \item 坡印廷矢量:$\mathbf{S} = \mathbf{E} \times \mathbf{H}$
    \item 动量密度:$\mathbf{g} = \varepsilon \mathbf{E} \times \mathbf{B}$
\end{itemize}

\subsection{平面电磁波的传播}

\subsubsection{理想介质}
\begin{itemize}
    \item 电场、磁场、传播方向之间相互垂直,是横电磁波
    \item 无衰减,电场与磁场振幅不变
    \item 波阻抗为实数,电场与磁场同相位
    \item 电磁波的相速度与频率无关,即无色散
    \item 电场能量密度等于磁场能量密度
    \item 能量的传输速度等于相速度
\end{itemize}

\subsubsection{导电媒质}

特征:电导率$\sigma \neq 0$,会引发传导电流$J=\sigma E$

\textbf{复介电常数方法}

定义一个复介电常数
\begin{align}
    \varepsilon_c=\varepsilon + i\frac{\sigma}{\omega}
\end{align}
对应地得到复的波数:
\begin{align}
    k_c=\omega \sqrt{\mu \varepsilon_c}
\end{align}
并引入衰减常数$\alpha[Np/m]$、相位常数$\beta[rad/m]$和电磁波的传播常数$\gamma[1/m]$:
\begin{align}
    k_c=i\gamma=\beta+i\alpha
\end{align}

导电媒质中的波动方程:
\begin{align}
    &\nabla^2\mathbf{E}+k_c^2\mathbf{E}=0\\
    &\nabla^2\mathbf{H}+k_c^2\mathbf{H}=0\\
    &k_c^2=\omega^2\mu\varepsilon_c=\omega^2\mu\varepsilon+i\omega\mu\sigma
\end{align}
平面波解:
\begin{align}
    \dot{\mathbf{E}}(z,t)=&E_{xm} e^{-\alpha z}e^{i \beta z}\hat{\mathbf{x}}\\
    \dot{\mathbf{H}}(z,t)=&\frac{1}{|\eta_c|}e^{i\phi}E_{xm} e^{-\alpha z}e^{i \beta z}\hat{\mathbf{y}}
\end{align}
注意,此时电磁场不再同相位:导电媒质本征阻抗此时为复数,\textbf{电场超前磁场相位}$\phi$:
\begin{align}
    \eta_c=\sqrt{\frac{\mu}{\varepsilon_c}}=|\eta_c|e^{-i\phi}
\end{align}


\textbf{平面波的传播特性}
\begin{itemize}
    \item 三个传播参数:
    
    由$k_c^2=\omega^2\mu\varepsilon+i\omega\mu\sigma$,可解出$\alpha$和$\beta$:

    \begin{align}
        \alpha=&\omega\sqrt{\frac{\mu\varepsilon}{2}}\sqrt{\sqrt{1+\left(\frac{\sigma}{\omega\varepsilon}\right)^2}-1}\\
        \beta=&\omega\sqrt{\frac{\mu\varepsilon}{2}}\sqrt{\sqrt{1+\left(\frac{\sigma}{\omega\varepsilon}\right)^2}+1}
    \end{align}
    此时$\alpha$,$\beta$和$\gamma$均与频率有关,表现出色散特性
    \item 相速度:与频率有关,是色散波。一般来讲,导电系数越大,色散越强。
    \begin{align}
        v_p=\frac{\omega}{\beta(\omega)}
    \end{align}
    \item 电磁场关系
    
    (1)仍通过波阻抗联系:
    \begin{align}
        \mathbf{E} =& \eta_c \cdot \mathbf{H} \times \hat{\mathbf{n}}=|\eta_c|e^{-i\phi}\cdot \mathbf{H} \times \hat{\mathbf{n}} \\
        \mathbf{H} =& \frac{1}{\eta_c} \cdot \hat{\mathbf{n}} \times \mathbf{E}=\frac{1}{|\eta_c|}e^{i\phi} \cdot \hat{\mathbf{n}} \times \mathbf{E}
    \end{align}

    (2)E、H、n三者仍互相垂直,构成右手系
    
    (3)电场超前相位$\phi=\dfrac{1}{2}\arctan \dfrac{\sigma}{\omega\varepsilon}$
    \begin{align}
        \eta_c=|\eta_c|e^{-i\frac{1}{2}\arctan \frac{\sigma}{\omega\varepsilon}}
    \end{align}

    \item 能量密度
    
    
    \begin{align}
        w_e=&\frac{1}{2}\varepsilon |E|^2=\frac{\varepsilon}{2}E_{xm}^2  e^{-2\alpha z}  \cos^2(\omega t - \beta z)\\
        w_m=&\frac{1}{2}\mu |H|^2=\frac{\mu}{2}\frac{E_{xm}^2}{|\eta_c|^2} e^{-2\alpha z} \cos^2(\omega t - \beta z + \phi)\\
        =&\frac{\varepsilon}{2}E_{xm}^2  e^{-2\alpha z}  \cos^2(\omega t - \beta z) \left[ 1+\left(\frac{\sigma}{\omega\varepsilon}\right)^2\right] ^{1/2}\\
        =& w_e   \left[ 1+\left(\frac{\sigma}{\omega\varepsilon}\right)^2\right] ^{1/2}
    \end{align}

    结论:导电媒质中均匀平面波的磁场能量大于电场能量

    \item 能流密度:衰减快于场的衰减
    \begin{align}
        \mathbf{S}_{ave}=\frac{1}{2}\mathrm{Re}[\dot{\mathbf{E}}\times \dot{\mathbf{H}}^*]=&\frac{1}{2} \frac{E_{xm}^2}{|\eta_c|} e^{-2\alpha z} \cos \phi\, \hat{\mathbf{n}}\\
    \end{align}
\end{itemize}

\textcolor{red}{良导体近似}

在良导体中,$\sigma \gg \omega \varepsilon$,则有近似关系:
\begin{align}
    \alpha \approx& \beta \approx \sqrt{\frac{\omega \mu \sigma}{2}}=\sqrt{\pi f\mu\sigma}\\
\end{align}

于是传播性质简化为:
\begin{itemize}
    \item 相速度:$v_p=\dfrac{\omega}{\beta}=\sqrt{\dfrac{2\omega}{\mu\sigma}}=\sqrt{\dfrac{4\pi f}{\mu\sigma}}$
    \item 波长:$\lambda=\dfrac{2\pi}{\beta}=2\sqrt{\dfrac{\pi}{f\mu\sigma}}=2\pi\sqrt{\dfrac{2}{\omega \mu\sigma}}$
    \item 波阻抗:$|\eta_c|=\sqrt{\dfrac{\omega \mu}{2\sigma}}$
    \item 电磁场相位差:$\phi=\dfrac{\pi}{4}$
    \item 能量密度:$w_m \approx w_e \dfrac{\sigma}{\omega \varepsilon} \gg w_e$
    \item 能流密度:$\mathbf{S}_{ave}=\dfrac{1}{2} \frac{E_{xm}^2}{|\eta_c|} e^{-2\alpha z} \dfrac{\sqrt{2}}{2}\, \hat{\mathbf{n}}$
\end{itemize}

\textbf{\textcolor{blue}{重要概念:趋肤深度}}

定义:电磁波穿入良导体时,当波的幅度下降为表面处振幅的$1/e$时,波在良导体中传播的距离

\begin{align}
    \delta=\frac{1}{\alpha}=\sqrt{\frac{2}{\omega \mu \sigma}}=\sqrt{\frac{1}{\pi f \mu \sigma}}
\end{align}

对良导体,有$\delta = 1/\alpha = 1/\beta = \lambda / 2\pi$

\textcolor{red}{弱导体近似}
\begin{align}
    \alpha \approx& \frac{\sigma}{2} \sqrt{\frac{\mu}{\varepsilon}}\\
    \beta \approx& \omega \sqrt{\mu \varepsilon}\\
    \eta_c\approx & \sqrt{\frac{\mu}{\varepsilon}}\left(1 + i \frac{\sigma}{2\omega\varepsilon}\right)
\end{align}
\begin{itemize}
    \item 衰减较小
    \item 相位常数和非导电媒质中基本相等
    \item 电场和磁场之间相位差很小
\end{itemize}


 \section{电磁波的反射和折射}

 计算框架:
 \begin{itemize}
    \item 分区写出入射波、反射波、透射波各分量的场表达式
    \item 应用边界条件(不同区域的k的关系、幅度、相位关系),列出方程
    \item 解方程,得到反射系数和透射系数
 \end{itemize}

 \subsection{边界条件}
    (1) 场强条件:
    
        \qquad (a)电场的切向分量、磁场的法向分量在边界上连续;

        \qquad (b)对于理想介质交界面,电场的法向分量、磁场的切向分量在边界上均连续,因为这一交界面上不存在面电流。

    \qquad (c) 需注意:区分极化方向;总的电磁场是所有波的叠加

    理想介质交界面条件:
    \begin{align}
        E_{ix}+E_{rx}=&E_{tx+}+E_{tx-}\\
        E_{iy}+E_{ry}=&E_{ty+}+E_{ty-}\\
        H_{ix}+H_{rx}=&H_{tx+}+H_{tx-}\\
        H_{iy}+H_{ry}=&H_{ty+}+H_{ty-}
    \end{align}

    (2) 相位匹配条件:所有波的波矢在边界面的切向分量相等

    需注意:是针对单个波而言,不用叠加;

    导出结论:Snell定律
    \begin{align}
        &k_1 \sin\theta_i = k_1 \sin\theta_r = k_2 \sin\theta_t\\
        &k=\omega \sqrt{\mu \varepsilon}\\
        \Rightarrow& \begin{cases}
            \theta_i=\theta_r\\
            \dfrac{\sin\theta_t}{\sin\theta_i}=\sqrt{\dfrac{\mu_1\varepsilon_1}{\mu_2\varepsilon_2}}=\dfrac{n_1}{n_2}
        \end{cases}
    \end{align}
    非磁性介质中,入射角/透射角满足
    \begin{align}
        \frac{\sin\theta_t}{\sin\theta_i}=\sqrt{\dfrac{\varepsilon_1}{\varepsilon_2}}
    \end{align}


\subsection{理想介质单一交界面的解}
\subsubsection{垂直极化:电场方向垂直于法平面,磁场方向由右手螺旋定则确定}
\begin{align}
    &E_{1y}=E_i+E_r=(1+\Gamma)E_0  \\
    &E_{2y}=E_{t}=\tau E_0\\
    &H_{1x}=H_{ix}+H_{rx}=\frac{1}{\eta_1}E_0(\cos\theta_i-\Gamma \cos\theta_r)
    =\frac{1}{\eta_1}E_0(1-\Gamma)\cos\theta_i\\
    &H_{2x}=H_{tx}=\frac{1}{\eta_2}\tau E_0 \cos\theta_t
\end{align}

透射、反射系数:
\begin{align}
    \Gamma_\perp=\frac{\eta_2\cos\theta_i-\eta_1\cos\theta_t}{\eta_2\cos\theta_i+\eta_1\cos\theta_t}\\
    \tau_\perp=\frac{2\eta_2\cos\theta_i}{\eta_2\cos\theta_i+\eta_1\cos\theta_t}
\end{align}

非磁性介质中,
\begin{align}
    \Gamma_\perp=&-\frac{\sin(\theta_i-\theta_t)}{\sin(\theta_i+\theta_t)}
    =\frac{\cos\theta_i-\sqrt{\dfrac{\varepsilon_2}{\varepsilon_1}-\sin^2\theta_i}}{\cos\theta_i+\sqrt{\dfrac{\varepsilon_2}{\varepsilon_1}-\sin^2\theta_i}}\\
    \tau_\perp=&\frac{2\cos\theta_i\sin\theta_t}{\sin(\theta_i+\theta_t)}=\frac{2\cos\theta_i}{\cos\theta_i+\sqrt{\dfrac{\varepsilon_2}{\varepsilon_1}-\sin^2\theta_i}}
\end{align}

反射-透射关系:
\begin{align}
    &1+\Gamma_\perp=\tau_\perp \\
    &1-\Gamma_\perp^2=\frac{\eta_1\cos\theta_t}{\eta_2\cos\theta_i}\tau_\perp^2
\end{align}

\subsubsection{水平极化:磁场方向垂直于法平面,电场方向由右手螺旋定则确定}
\begin{align}
    E_{1x}=&E_i+E_r=E_0(1-\Gamma)\cos\theta_i  \\
    E_{2x}=&E_{tx}=\tau E_0\cos\theta_t\\\
    H_{1y}=&H_i+H_r=\frac{1}{\eta_1}E_0(1+\Gamma) \\
    H_{2y}=&H_{ty}=\frac{1}{\eta_2}\tau E_0
\end{align}

透射、反射系数:
\begin{align}
    \Gamma_\parallel=&\frac{\eta_1\cos\theta_i-\eta_2\cos\theta_t}{\eta_1\cos\theta_i+\eta_2\cos\theta_t}\\
    \tau_\parallel=&\frac{2\eta_2\cos\theta_i}{\eta_1\cos\theta_i+\eta_2\cos\theta_t}
\end{align}

非磁性介质:
\begin{align}
    \Gamma_\parallel=&\frac{\tan(\theta_i-\theta_t)}{\tan(\theta_i+\theta_t)}
    =\dfrac{\dfrac{\varepsilon_2}{\varepsilon_1}\cos\theta_i-\sqrt{\dfrac{\varepsilon_2}{\varepsilon_1}-\sin^2\theta_i}}{\dfrac{\varepsilon_2}{\varepsilon_1}\cos\theta_i+\sqrt{\dfrac{\varepsilon_2}{\varepsilon_1}-\sin^2\theta_i}}\\
    \tau_\parallel=&\frac{2\cos\theta_i\sin\theta_t}{\sin(\theta_i+\theta_t)\cos(\theta_i-\theta_t)}=\frac{2\sqrt{\dfrac{\varepsilon_2}{\varepsilon_1}}\cos\theta_i}{\dfrac{\varepsilon_2}{\varepsilon_1}\cos\theta_i+\sqrt{\dfrac{\varepsilon_2}{\varepsilon_1}-\sin^2\theta_i}}
\end{align}

反射-透射关系:
\begin{align}
    &1+\Gamma_\parallel=\frac{\eta_1}{\eta_2}\tau_\parallel \\
    &1-\Gamma_\parallel^2=\tau_\parallel^2
\end{align}

\subsection{全透射与全反射}
\subsubsection{全透射:反射系数为0,仅出现在平行极化分量中}
条件:
\begin{align}
    \sin\theta_i=\sqrt{\frac{\varepsilon_2}{\varepsilon_1+\varepsilon_2}}\\
    \cos\theta_i=\sqrt{\frac{\varepsilon_1}{\varepsilon_1+\varepsilon_2}}
\end{align}
临界角度:Brewster角
\begin{align}
    \theta_B=\arctan\sqrt{\frac{\varepsilon_1}{\varepsilon_2}}=\arctan\frac{n_2}{n_1}
\end{align}
性质:
\begin{align}
    \theta_B+\theta_t=\frac{\pi}{2}
\end{align}

\subsubsection{全反射:透射系数为0,各种极化模式都可能发生}
条件:
\begin{align}
    &\sin\theta_i=\sqrt{\frac{\varepsilon_2}{\varepsilon_1}}\\
    &\theta_i=\arcsin\sqrt{\frac{\varepsilon_2}{\varepsilon_1}}=\theta_c
\end{align}
即如下两条:
\begin{itemize}
    \item 入射波媒质1向媒质2斜入射,且$\varepsilon_1>\varepsilon_2$
    \item 入射角$\theta_i$大于临界角$\theta_c$,即$\theta_c\leq \theta_i \leq 90^\circ$
\end{itemize}

此时透射波可写为衰减形式:
\begin{align}
    \mathbf{E}_t=\mathbf{E}_{t0}    e^{-\alpha z}e^{i\beta x}
\end{align}
其中
\begin{align}
    &\alpha=k_2\sqrt{\left(\frac{\sin\theta_i}{\sin\theta_c}\right)^2-1}\\
    &\beta=k_2\frac{\sin\theta_i}{\sin\theta_c}
\end{align}
衰减长度$\delta=k_1\sqrt{\sin^2\theta_i-\sin^2\theta_c}$

Note:衰减的透射波会引发Goos-Hänchen位移现象。


\section{波导基本理论}

\subsection{纵向均匀系统求解思路}
\begin{itemize}
    \item 第一步:求$E_z(x,y)$和$B_z(x,y)$的分布,满足横向Laplace方程或Helmholtz方程及边界条件
    \begin{align}
        (\nabla_t^2+\omega^2\mu\varepsilon-k_z^2)E_z=0\\
        (\nabla_t^2+\omega^2\mu\varepsilon-k_z^2)B_z=0\\
        k_z^2=\omega^2\mu\varepsilon - k_c^2=k_0^2-k_c^2\\
    \end{align}
    \item 代入波导壁的边界条件,求解纵向场
    \item 代入纵横关系得到所有的场分量
\end{itemize}


\subsection{从麦克斯韦方程组导出的波动方程}
Helmholtz方程:
\begin{align}
    \nabla^2 \mathbf{E} + k^2 \mathbf{E} = 0 \\
    \nabla^2 \mathbf{H} + k^2 \mathbf{H} = 0
\end{align}
式中,$k = \omega \sqrt{\mu\varepsilon}$为波数。

\subsubsection{直角坐标系的波动方程及其基本解}


\subsubsection{柱坐标系的波动方程及其基本解}


\subsection{纵向均匀系统的纵横关系}


考虑沿z方向均匀的波导,电磁场可表示为:
\begin{align}
    \mathbf{E}(x,y,z)=\mathbf{E}_t(x,y)e^{-ik_z z}+ \hat{\mathbf{z}} E_z(x,y)e^{-ik_z z}\\
    \mathbf{H}(x,y,z)=\mathbf{H}_t(x,y)e^{-ik_z z}+ \hat{\mathbf{z}} H_z(x,y)e^{-ik_z z}
\end{align}
从Helmholtz方程出发,得到横向场与纵向场的关系:
\begin{align}
    \mathbf{E}_t=&\frac{i}{\mu\varepsilon\omega^2-k_z^2}\left( k_z \nabla_t E_z - \omega \mu \hat{\mathbf{z}} \times \nabla_t H_z \right)\\
    \mathbf{H}_t=&\frac{i}{\mu\varepsilon\omega^2-k_z^2}\left( k_z \nabla_t H_z  + \omega \varepsilon \hat{\mathbf{z}} \times \nabla_t E_z \right)
\end{align}

或者磁场用B表示:
\begin{align}
    \mathbf{E}_t=&\frac{i}{\mu\varepsilon\omega^2-k_z^2}\left( k_z \nabla_t E_z - \omega \hat{\mathbf{z}} \times \nabla_t B_z \right)\\
    \mathbf{B}_t=&\frac{i}{\mu\varepsilon\omega^2-k_z^2}\left( k_z \nabla_t B_z  + \omega \mu \varepsilon \hat{\mathbf{z}} \times \nabla_t E_z \right)
\end{align}

反之,有
\begin{align}
    \nabla_t \mathbf{E}_t=-ik_z E_z\\
    \nabla_t \mathbf{H}_t=-ik_z H_z
\end{align}

\subsection{纵向均匀系统的模式}
\begin{itemize}
    \item TEM模:$E_z=0,H_z=0$,横向电磁波,不能在单导体波导中存在
    \item TE模:$E_z=0,H_z\neq0$,横电波
    \item TM模:$H_z=0,E_z\neq0$,横磁波
\end{itemize}

\subsection{模式的传输特性}
\textbf{1、传输条件}:对于一个给定的几何结构,该结构会决定一组截止参数,$f_c,\,\lambda_c,\,k_c$,只有当工作频率$f>f_c$时,该模式才能传输。

即:
\begin{align}
    \lambda<\lambda_c,\,f>f_c,\Rightarrow \beta^2=k^2-k_c^2>0
\end{align}

\textbf{2、随激励频率变化的图像}:

(1)当激励频率刚刚超过截止频率时,在波导中激励起基模的波;

(2)随着激励频率增加,横向波数不变,纵向波数增大,导波波长减小,相速度减小。即,一个
“方块”横向不变,纵向长度减小。

(3)激励频率大于某个高阶模式的截止频率时,该高阶模式也会被激励起来,但低阶模式依旧存在。

\textbf{3、传输条件相关的物理参量}:
\begin{itemize}
    \item 相移常数/纵向波数$k_z$或$\beta$:
    \begin{align}
        \beta=\sqrt{k^2-k_c^2}=k\sqrt{1-\left(\frac{k_c}{k}\right)^2}=k\sqrt{1-\left(\frac{\lambda}{\lambda_c}\right)^2}
    \end{align}
    \item 相速度$v_p$
    \begin{align}
        v_p=\frac{\omega}{\beta}=\frac{c}{\sqrt{1-\left(\dfrac{\lambda}{\lambda_c}\right)^2}}
    \end{align}
    \item 导波波长$\lambda_g$
    \begin{align}
        \lambda_g=\frac{2\pi}{\beta}=\frac{\lambda}{\sqrt{1-\left(\dfrac{\lambda}{\lambda_c}\right)^2}}
    \end{align}
    \item 群速度$v_g$
    \begin{align}
        v_g=\frac{\mathrm{d}\omega}{\mathrm{d}\beta}=c\sqrt{1-\left(\dfrac{\lambda}{\lambda_c}\right)^2}
    \end{align}
    \item 色散关系:双曲线,渐近线为$v=\pm c$
    \item 波阻抗:相互垂直的横向电场与横向磁场之比
    \begin{align}
        \eta=\frac{(\hat{\mathbf{z}}\times \mathbf{E}_t)_i}{(\mathbf{H}_t)_i},\quad i=x,y
    \end{align}
\end{itemize}

\textbf{4、截止状态下的衰减}

当$k<k_c$时,纵向波数$k_z$为虚数,记为$ik_z=\alpha$,则波在z方向上呈指数衰减:
\begin{align}
    &\alpha=\sqrt{k_c^2-k^2}=k\sqrt{\left(\frac{k_c}{k}\right)^2-1}=k\sqrt{\left(\frac{\lambda}{\lambda_c}\right)^2-1}\\
    &E(z),\,H(z)\propto e^{-\alpha z}\\
    &P(z)\propto e^{-2\alpha z}
\end{align}

截止状态与衰减行波的区别:截止状态是无相位变化的原地衰减,而衰减行波是有相位变化的传播衰减。

传播参数变化:$\lambda\rightarrow \lambda_c,\,v_p\rightarrow \infty,\,\lambda_g\rightarrow \infty,\,v_g\rightarrow 0$

\subsection{波导的阻抗与功率传输}
传输功率
\begin{align}
    P=&\begin{cases}
        \dfrac{\omega\mu k_z}{2k_c^2}\int_S |H_z|^2 \mathrm{d}S & \text{TE模}\\
        \dfrac{\omega\varepsilon k_z}{2k_c^2}\int_S |E_z|^2 \mathrm{d}S & \text{TM模} 
    \end{cases}
\end{align}
式中,$k_c^2=k^2-k_z^2$为截止波数。

功率衰减
\begin{align}
    -\frac{\mathrm{d}P}{\mathrm{d}z}=\frac{1}{2\sigma\delta}|\mathbf{H}_t|^2\mathrm{d}l
\end{align}
Note: $R_s=1/\sigma\delta$为表面电阻,$\delta=\sqrt{2/\omega \mu \sigma}$为趋肤深度。

衰减常数
\begin{align}
    \alpha=-\frac{1}{2P}\frac{\mathrm{d}P}{\mathrm{d}z}
\end{align}


\section{几种常见波导}

\subsection{矩形波导}

结构:长边a,短边b的矩形波导。

传播参数:
\begin{itemize}
    \item 纵向波数:$k_z$或$\beta$,$\beta$又称为相移常数。
    \item 截止波数:$k_c=\sqrt{\left(\dfrac{m\pi}{a}\right)^2+\left(\dfrac{n\pi}{b}\right)^2}$
    \item 横向波数:$k_x =\dfrac{m\pi}{a},k_y=\dfrac{n\pi}{b}$
    \item 关系:$k_x^2+k_y^2+k_z^2=k^2=\omega^2\mu\varepsilon$
\end{itemize}

边界条件:
\begin{align}
    \text{(常用)壁上电场切向分量为0:}\hat{\mathbf{n}}\times \mathbf{E}|_s=0\\
    \text{磁场法向分量为0:}\hat{\mathbf{n}}\cdot \mathbf{B}|_s=0
\end{align}

具体的纵向场边界条件:
\begin{align}
    \text{TE模:}&\frac{\partial H_z}{\partial n}\bigg|_s=0\\
    \text{TM模:}&E_z\big|_s=0
\end{align}

\subsubsection{基础性质}
\begin{itemize}
    \item 截止频率$f_c=\dfrac{c}{2}\sqrt{\left(\dfrac{m}{a}\right)^2+\left(\dfrac{n}{b}\right)^2}$
    \item 导波波长$\lambda=\dfrac{\lambda_0}{\sqrt{1-\left(\dfrac{\lambda_0}{\lambda_c}\right)^2}}$
    \item 相速度$v_p=\dfrac{c}{\sqrt{1-\left(\dfrac{f_c}{f}\right)^2}}$
    \item 群速度$v_g=c\sqrt{1-\left(\dfrac{f_c}{f}\right)^2}$
    \item 波阻抗
    \begin{align}
        \eta_{TE}=&\frac{\eta_0}{\sqrt{1-\left(\dfrac{f_c}{f}\right)^2}}\\
        \eta_{TM}=&\eta_0\sqrt{1-\left(\dfrac{f_c}{f}\right)^2}
    \end{align}
    \item 传输功率
    \begin{align}
        P_{TE}=&\frac{\omega\mu k_z ab}{4\epsilon_{mn}k_c^2}|H_{mn}|^2\\
        P_{TM}=&\frac{\omega\varepsilon k_z ab}{8k_c^2}|E_{mn}|^2
    \end{align}
    式中,$\epsilon_{mn}=\begin{cases}
        1,&mn=0\\
        2,&mn\neq 0
    \end{cases}$
    \item 损耗常数
    \begin{align}
        \text{常用:}\alpha_{TE10}=&\frac{2R_s}{b\eta\sqrt{1-(f_c/f)^2}}\left[ 1+\frac{2b}{a}\left(\frac{f_c}{f}\right)^2  \right]\\
        \text{任意mn模式:}\alpha_{TE/Mmn}=&\frac{2R_s}{b\eta\sqrt{1-(f_c/f)^2}}\frac{m^2b^3+n^2a^3}{m^2ab^2+n^2a^3}
    \end{align}
\end{itemize}

%\newpage
\subsubsection{场分布表达式:}
\textbf{TE模:最低模式TE10}
\begin{align}
    E_z=&0\\
    H_z=&H_0 \cos\left(\frac{m\pi x}{a}\right)\cos\left(\frac{n\pi y}{b}\right)e^{ik_z z}\\
    E_x=&-\frac{i\omega \mu}{k_c^2}\frac{n\pi}{b}H_0 \cos\left(\frac{m\pi x}{a}\right)\sin\left(\frac{n\pi y}{b}\right)e^{ik_z z}\\
    E_y=&\frac{i\omega \mu}{k_c^2}\frac{m\pi}{a}H_0 \sin\left(\frac{m\pi x}{a}\right)\cos\left(\frac{n\pi y}{b}\right)e^{ik_z z}\\
    H_x=&-\frac{ik_z}{k_c^2}\frac{m\pi}{a }H_0 \sin\left(\frac{m\pi x}{a}\right)\cos\left(\frac{n\pi y}{b}\right)e^{ik_z z}\\
    H_y=&-\frac{ik_z}{k_c^2}\frac{n\pi}{b}H_0 \cos\left(\frac{m\pi x}{a}\right)\sin\left(\frac{n\pi y}{b}\right)e^{ik_z z}
\end{align}

\textbf{TM模:最低模式TM11}
\begin{align}
    H_z=&0\\
    E_z=&E_0 \sin\left(\frac{m\pi x}{a}\right)\sin\left(\frac{n\pi y}{b}\right)e^{ik_z z}\\
    E_x=&\frac{ik_z}{k_c^2}\frac{m\pi}{a}E_0 \cos\left(\frac{m\pi x}{a}\right)\sin\left(\frac{n\pi y}{b}\right)e^{ik_z z}\\
    E_y=&\frac{ik_z}{k_c^2}\frac{n\pi}{b}E_0 \sin\left(\frac{m\pi x}{a}\right)\cos\left(\frac{n\pi y}{b}\right)e^{ik_z z}\\
    H_x=&-\frac{i\omega\varepsilon}{k_c^2}\frac{n\pi}{b}E_0 \sin\left(\frac{m\pi x}{a}\right)\cos\left(\frac{n\pi y}{b}\right)e^{ik_z z}\\
    H_y=&\frac{i\omega\varepsilon}{k_c^2}\frac{m\pi}{a}E_0 \cos\left(\frac{m\pi x}{a}\right)\sin\left(\frac{n\pi y}{b}\right)e^{ik_z z}
\end{align}  


\newpage
\subsubsection{场分布图}

\textbf{TM11}

\begin{figure}[H]
    \centering
    \includegraphics[width=0.6\textwidth]{./Figures/Field_TM11.png}
    %\caption{矩形波导TM11模式场分布图}
\end{figure}

\textbf{TM32}
\begin{figure}[H]
    \centering
    \includegraphics[width=0.6\textwidth]{./Figures/Field_TM32.png}
    %\caption{矩形波导TM32模式场分布图}
\end{figure}

\textbf{TE01}
\begin{figure}[H]
    \centering
    \includegraphics[width=0.6\textwidth]{./Figures/Field_TE01.png}
    %\caption{矩形波导TE01模式场分布图}
\end{figure}


\newpage
\textbf{TE10}
\begin{figure}[H]
    \centering
    \includegraphics[width=0.6\textwidth]{./Figures/Field_TE10.png}
    %\caption{矩形波导TE10模式场分布图}
\end{figure}

\textbf{TE30}
\begin{figure}[H]
    \centering
    \includegraphics[width=0.6\textwidth]{./Figures/Field_TE30.png}
    %\caption{矩形波导TE30模式场分布图}
\end{figure}


\textbf{TE11}
\begin{figure}[H]
    \centering
    \includegraphics[width=0.6\textwidth]{./Figures/Field_TE11.png}
    %\caption{矩形波导TE11模式场分布图}
\end{figure}


\newpage
\textbf{TE32}
\begin{figure}[H]
    \centering
    \includegraphics[width=0.6\textwidth]{./Figures/Field_TE32.png}
    %\caption{矩形波导TE32模式场分布图}
\end{figure}



% \subsubsection{矩形波导的模式}

% \subsubsection{截止频率和传播特性}
% \subsubsection{TE模式场分布}
% \subsubsection{TM模式场分布}




% \subsubsection{波导的阻抗和功率传输}

\subsection{圆波导}

结构:半径为$a$的金属圆柱波导,采用圆柱坐标$(r,\phi,z)$描述场分布。

传播参数:
\begin{itemize}
    \item 纵向波数:$k_z=\beta=\sqrt{k^2-k_c^2}$,其中$k=\omega\sqrt{\mu\varepsilon}$。
    \item 截止波数:$k_{c,\mathrm{TE}_{mn}}=\dfrac{\chi'_{mn}}{a}$,$k_{c,\mathrm{TM}_{mn}}=\dfrac{\chi_{mn}}{a}$,$\chi'_{mn}$为$J_m'(x)$的第$n$个零点,$\chi_{mn}$为$J_m(x)$的第$n$个零点。
    \item 横向波数:$k_r=k_c$,径向分布满足Bessel方程,角向分布为$\cos(m\phi)$或$\sin(m\phi)$形式。
    \item 关系:$k_r^2+k_z^2=k^2$,且$k_c^2=k_r^2=\left(\dfrac{\chi}{a}\right)^2$。
\end{itemize}

边界条件:
\begin{align}
    	\text{壁上电场切向分量为0:}&\hat{\mathbf{n}}\times \mathbf{E}|_{r=a}=0\\
    	\text{壁上磁场法向分量为0:}&\hat{\mathbf{n}}\cdot \mathbf{B}|_{r=a}=0
\end{align}

具体的纵向场边界条件:
\begin{align}
    	\text{TE模:}&\left.\frac{\partial H_z}{\partial r}\right|_{r=a}=0\Rightarrow J_m'(\chi'_{mn})=0\\
    	\text{TM模:}&E_z\big|_{r=a}=0\Rightarrow J_m(\chi_{mn})=0
\end{align}

\subsubsection{基础性质}
\begin{itemize}
    \item 截止频率:$f_{c,\mathrm{TE}_{mn}}=\dfrac{c}{2\pi} \dfrac{\chi'_{mn}}{a}$,$f_{c,\mathrm{TM}_{mn}}=\dfrac{c}{2\pi} \dfrac{\chi_{mn}}{a}$。
    \item 导波波长:$\lambda_g=\dfrac{\lambda_0}{\sqrt{1-\left(\dfrac{f_c}{f}\right)^2}}$,$\lambda_0=\dfrac{2\pi}{k}$。
    \item 相速度:$v_p=\dfrac{c}{\sqrt{1-\left(\dfrac{f_c}{f}\right)^2}}$。
    \item 群速度:$v_g=c\sqrt{1-\left(\dfrac{f_c}{f}\right)^2}$。
    \item 波阻抗:
    \begin{align}
        \eta_{TE}=&\frac{\eta_0}{\sqrt{1-\left(\dfrac{f_c}{f}\right)^2}}\\
        \eta_{TM}=&\eta_0\sqrt{1-\left(\dfrac{f_c}{f}\right)^2}
    \end{align}
    \item 传输功率(以$H_z=H_0J_m\left(\dfrac{\chi'_{mn} r}{a}\right)\cos m\phi\,e^{ik_z z}$、$E_z=E_0J_m\left(\dfrac{\chi_{mn} r}{a}\right)\cos m\phi\,e^{ik_z z}$为例):
    \begin{align}
        P_{TE}=&\frac{\omega\mu\beta}{2k_c^2}|H_0|^2\int_0^{2\pi}\int_0^{a}\Big|J_m\Big(\frac{\chi'_{mn} r}{a}\Big)\cos m\phi\Big|^2 r\,\mathrm{d}r\mathrm{d}\phi\\
        P_{TM}=&\frac{\omega\varepsilon\beta}{2k_c^2}|E_0|^2\int_0^{2\pi}\int_0^{a}\Big|J_m\Big(\frac{\chi_{mn} r}{a}\Big)\cos m\phi\Big|^2 r\,\mathrm{d}r\mathrm{d}\phi
    \end{align}
    其中积分可借助$\int_0^1 J_m^2(\alpha x) x\,\mathrm{d}x$的标准结果化为闭式表达。
    \item 损耗常数(金属壁面电导率$\sigma$):
    \begin{align}
        \alpha_{TE}=&\frac{R_s}{2P_{TE}}\int_0^{2\pi}\left(\big|H_\phi(r=a,\phi)\big|^2+\big|H_\phi(r=a,\phi)\big|^2\right) a\,\mathrm{d}\phi\\
        \alpha_{TM}=&\frac{R_s}{2P_{TM}}\int_0^{2\pi}\big|H_\phi(r=a,\phi)\big|^2 a\,\mathrm{d}\phi,
    \end{align}
    $R_s=\sqrt{\dfrac{\omega\mu}{2\sigma}}$为表面电阻。
\end{itemize}

传输功率和损耗常数的具体表达式:
\begin{align}
    % TM 模式传输功率
&P_{\mathrm{TM}} = \frac{\pi a^2 \omega \varepsilon \beta}{2 \epsilon_n k_c^2} |E_0|^2 J_{n+1}^2(\chi_{ni})\\
% TM 模式损耗常数
&\alpha_{\mathrm{TM}} = \frac{R_s}{a \eta \sqrt{1 - (f_c/f)^2}}\\
% TE 模式传输功率
&P_{\mathrm{TE}} = \frac{\pi a^2 \omega \mu \beta}{2 \epsilon_n k_c^2} |H_0|^2 \left( 1 - \frac{n^2}{\chi_{ni}'^2} \right) J_n^2(\chi'_{ni})\\
% TE 模式损耗常数
&\alpha_{\mathrm{TE}} = \frac{R_s}{a \eta \sqrt{1 - (f_c/f)^2}} \left[ \left( \frac{f_c}{f} \right)^2 + \frac{n^2}{\chi_{ni}'^2 - n^2} \right]
\end{align}
式中,
\begin{align}
    \epsilon_n=\begin{cases}
        2 & n=0 \\
        1 & n\neq 0
    \end{cases}
\end{align}

%\newpage
\subsubsection{场分布表达式:}
	\textbf{TE模:最低模式$\mathrm{TE}_{11}$,$k_c=\chi'_{ni}/a$}

    偶模式:

\begin{align}
    H_z    &= H_0 J_n(k_c r) \cos(n\phi) e^{jk_z z} \\
    H_r    &= \frac{ik_z}{k_c} H_0 J_n'(k_c r) \cos(n\phi) e^{jk_z z} \\
    H_\phi &= -\frac{ik_z n}{k_c^2 r} H_0 J_n(k_c r) \sin(n\phi) e^{jk_z z} \\
    E_r    &= -\frac{i\omega\mu n}{k_c^2 r} H_0 J_n(k_c r) \sin(n\phi) e^{jk_z z} \\
    E_\phi &= -\frac{i\omega\mu}{k_c} H_0 J_n'(k_c r) \cos(n\phi) e^{jk_z z}
\end{align}

奇模式:

\begin{align}
    H_z    &= H_0 J_n(k_c r) \sin(n\phi) e^{jk_z z} \\
    H_r    &= \frac{ik_z}{k_c} H_0 J_n'(k_c r) \sin(n\phi) e^{jk_z z} \\
    H_\phi &= \frac{ik_z n}{k_c^2 r} H_0 J_n(k_c r) \cos(n\phi) e^{jk_z z} \\
    E_r    &= \frac{i\omega\mu n}{k_c^2 r} H_0 J_n(k_c r) \cos(n\phi) e^{jk_z z} \\
    E_\phi &= -\frac{i\omega\mu}{k_c} H_0 J_n'(k_c r) \sin(n\phi) e^{jk_z z}
\end{align}


	\textbf{TM模:最低模式$\mathrm{TM}_{01}$,$k_c=\chi_{ni}/a$}

    偶模式:
\begin{align}
    H_z   &= 0 \\
    E_z    &= E_0 J_n(k_c r) \cos(n\phi) e^{jk_z z} \\
    E_r    &= \frac{ik_z}{k_c} E_0 J_n'(k_c r) \cos(n\phi) e^{jk_z z} \\
    E_\phi &= -\frac{ik_z n}{k_c^2 r} E_0 J_n(k_c r) \sin(n\phi) e^{jk_z z} \\
    H_r    &= \frac{i\omega\varepsilon n}{k_c^2 r} E_0 J_n(k_c r) \sin(n\phi) e^{jk_z z} \\
    H_\phi &= \frac{i\omega\varepsilon}{k_c} E_0 J_n'(k_c r) \cos(n\phi) e^{jk_z z}
\end{align}

奇模式:
\begin{align}
    H_z   &= 0 \\
    E_z    &= E_0 J_n(k_c r) \sin(n\phi) e^{ik_z z} \\
    E_r    &= \frac{ik_z}{k_c} E_0 J_n'(k_c r) \sin(n\phi) e^{ik_z z} \\
    E_\phi &= \frac{ik_z n}{k_c^2 r} E_0 J_n(k_c r) \cos(n\phi) e^{ik_z z} \\
    H_r    &= -\frac{i\omega\varepsilon n}{k_c^2 r} E_0 J_n(k_c r) \cos(n\phi) e^{ik_z z} \\
    H_\phi &= \frac{i\omega\varepsilon}{k_c} E_0 J_n'(k_c r) \sin(n\phi) e^{ik_z z}
\end{align}

\subsubsection{场分布图}
绘图规律:

$TE_{ni}$模式:
\begin{itemize}
    \item 若n为0,则电场线为同心圆,磁场线为径向线。沿径向,电场线的方向变化i-1次。
    \item 若n不为0,整个圆周分为2n个扇区;
    \item 每个扇区中,电场线绕成i-1个闭合环路,在壁面上有半个环路;任意两个相邻环路的旋转方向相反(即不产生相邻但反向的电场线)
    \item 磁场线连接各相邻电场线环路的中心,但不会跨过扇区分界线。
\end{itemize}

$TM_{ni}$模式:
\begin{itemize}
    \item 若n为0,则磁场线为同心圆,电场线为径向线。沿径向,磁场线的方向变换i-1次。
    \item 若n不为0,整个圆周分为2n个扇区;
    \item 每个扇区中,磁场线绕成i个闭合环路,任意两个相邻环路的旋转方向相反,即不产生相邻但反向的磁场线。
    \item 电场线连接各相邻磁场线环路的中心,但不会跨过扇区分界线。
\end{itemize}






\subsection{同轴线}
同轴线中多为TEM模式
\begin{itemize}
    \item 截止频率 $f_c=0$
    \item 波速 $v_p=v_g=c$
    \item 波阻抗 $\eta=\dfrac{\eta_0}{2\pi}\ln\dfrac{b}{a}$
    \item 传输功率 $P=\dfrac{E_m^2}{2\eta}\cdot 2\pi \left( b - a \right)$
    \item 损耗常数 $\alpha=\dfrac{R_s}{2\pi \eta} \left( \dfrac{1}{a} + \dfrac{1}{b} \right)$  
\end{itemize}

\subsection{介质加载波导}

\subsubsection{窄边单侧填充方波导}
结构:长$x=0\sim a$,宽$y=0\sim b$的矩形波导,其中长边$x=0\sim h$区域填充介质,$h<a$。

传播参数:
\begin{itemize}
    \item 两个区域各自有一个横向波数,但纵向波数相同。即有条件
    \begin{align}
        k_{x1}^2+k_y^2+k_z^2=k_1^2+k_z^2=k^2=\omega^2\mu_1\varepsilon_1\\
        k_{x2}^2+k_y^2+k_z^2=k_2^2+k_z^2=k^2=\omega^2\mu_2\varepsilon_2
    \end{align}
    \item y方向边界条件决定:$k_y=\dfrac{n\pi}{b}$,$n=0,1,2,\ldots$
    \item 在$k_y\neq 0$时的横向波数表示为:
    \begin{align}
        k_{xi}=\sqrt{\omega_c^2\varepsilon_i\mu_i-\left(\frac{n\pi}{b}\right)^2},\quad i=1,2
    \end{align}
    式中,$\omega_c$为截止角频率。
\end{itemize}

模式分析:

(1)$TE$模式:$E_x=0$,即$k_y=0$,仅允许$TE_{m0}$模式存在。

TE模式截止波数:需要求解以下方程决定:
    \begin{align}
        \text{电场交界面:}&\frac{\mu_1k_{x1}A}{k_1^2-k_z^2}\sin (k_{x1}h)=\frac{\mu_2k_{x2}B}{k_2^2-k_z^2}\sin[k_{x2}(h-a)]\\
        \text{磁场交界面:}&\frac{A}{k_1^2-k_z^2}\cos(k_{x1}h)=\frac{B}{k_2^2-k_z^2}\cos[k_{x2}(h-a)]\\
                          &A\cos(k_{x1}h)=B\cos[k_{x2}(h-a)]\\
        &k_{x1}^2-k_{x2}^2=\omega^2\mu_1\varepsilon_1-\omega^2\mu_2\varepsilon_2
    \end{align}
    仅允许$TE_{m0}$模式存在。

(2)$TM$模式:$H_x=0$,即$k_y=0$,由于$TM_{mn}$模式的n不能为0,介质加载矩形波导中不存在TM模式。

TM模式截止波数:需要求解以下方程决定:

原因:
    TM模式截止波数:需要求解以下方程决定:
    \begin{align}
        \text{磁场交界面:}&\frac{\varepsilon_1k_{x1}A}{k_1^2-k_z^2}\cos (k_{x1}h)=\frac{\varepsilon_2k_{x2}B}{k_2^2-k_z^2}\cos[k_{x2}(h-a)]\\
        \text{电场交界面:}&\frac{A}{k_1^2-k_z^2}\sin(k_{x1}h)=\frac{B}{k_2^2-k_z^2}\sin[k_{x2}(h-a)]\\
                          &A\sin(k_{x1}h)=B\sin[k_{x2}(h-a)]\\
        &k_{x1}^2-k_{x2}^2=\omega^2\mu_1\varepsilon_1-\omega^2\mu_2\varepsilon_2
    \end{align}
    由于$TM_{mn}$模式的n不能为0,介质加载矩形波导中不存在TM模式。

(3)x纵向模式:分为$E_y=0$的$TE^{(x)}$模式和$H_y=0$的$TM^{(x)}$模式。

计算要点:
\begin{itemize}
    \item x纵向的纵横关系:
    \begin{align}
        \mathbf{E}_{y,z}=\frac{1}{k^2-k_x^2}\left[  \frac{\partial }{\partial x}\nabla_{y,z}E_x -i\mu\omega \hat{\mathbf{x}}\times \nabla_{y,z}H_x \right]
        \mathbf{H}_{y,z}=\frac{1}{k^2-k_x^2}\left[  \frac{\partial }{\partial x}\nabla_{y,z}H_x +i\varepsilon\omega \hat{\mathbf{x}}\times \nabla_{y,z}E_x \right]
    \end{align}
    \item $TE^{(x)}$模式,交界面给出的截止条件:
    \begin{align}
        \frac{\mu_1A}{k_1^2-k_{x1}^2}\sin(k_{x1}h)=\frac{\mu_2B}{k_2^2-k_{x2}^2}\sin[k_{x2}(h-a)]\\
        \frac{k_{x1}A}{k_1^2-k_{x1}^2}\cos(k_{x1}h)=\frac{k_{x2}B}{k_2^2-k_{x2}^2}\cos[k_{x2}(h-a)]\\
        \frac{k_{x1}}{\mu_1}\cot(k_{x1}h)=\frac{k_{x2}}{\mu_2}\cot[k_{x2}(h-a)]\\
        \k_{x1}^2-k_{x2}^2=\omega^2\mu_1\varepsilon_1-\omega^2\mu_2\varepsilon_2
    \end{align}
    \item $TM^{(x)}$模式,交界面给出的截止条件:
    \begin{align}
        \frac{k_{x1}}{\varepsilon_1}\tan(k_{x1}h)=\frac{k_{x2}}{\varepsilon_2}\tan[k_{x2}(h-a)]\\
        k_{x1}^2-k_{x2}^2=\omega^2\mu_1\varepsilon_1-\omega^2\mu_2\varepsilon_2
    \end{align}
\end{itemize}

性质:
\begin{itemize}
    \item 对每个n,都有无穷多解,从小到大排序,序数即为m;
    \item 当$n=0$时,$TE_{m0}^{(x)}$ 和 $TE_{m0}^{(z)}$模式的截止频率相同;
    \item 对$TM^(x)$模式,仍然有$n\neq 0$.
\end{itemize}

(4)y纵向模式:分为$E_y=0$的$TE^{(y)}$模式和$H_y=0$的$TM^{(y)}$模式。

计算要点:
\begin{itemize}
    \item y纵向的纵横关系:
    \begin{align}
        \mathbf{E}_{z,x}=\frac{1}{k^2-k_y^2}\left[\frac{\partial}{\partial y}\nabla_{z,x}E_y -i\mu\omega \hat{\mathbf{y}}\times \nabla_{z,x}H_y \right]\\
        \mathbf{H}_{z,x}=\frac{1}{k^2-k_y^2}\left[\frac{\partial}{\partial y}\nabla_{z,x}H_y +i\varepsilon\omega \hat{\mathbf{y}}\times \nabla_{z,x}E_y \right]
    \end{align}
    \item $TE^{(y)}$模式,交界面给出的截止条件:\textcolor{red}{待补充}
    \item $TM^{(y)}$模式,交界面给出的截止条件:\textcolor{red}{待补充}
\end{itemize}



\subsubsection{双侧对称填充方波导}
不整理了,考出来就死给他看

\subsubsection{角向均匀填充圆波导}

\subsubsection{对称平板介质波导}

% \subsubsection{圆截面介质波导}

% \subsubsection{特征方程的求解}
% 不同边界条件下的特征方程。

% \subsection{介质加载圆波导和介质波导}

% \subsubsection{介质圆波导}
% \begin{itemize}
%     \item 介质波导的导模条件
%     \item 泄漏模和辐射模
% \end{itemize}

% \subsection{光纤原理}
% 光在光纤中的传播机制。

\section{谐振腔基础理论}

\subsection{波动方程及其基本解}

\subsection{谐振系统的纵横关系}

\subsubsection{波导的储能与品质因数}

\section{常见谐振腔}
\subsection{矩形谐振腔}
\begin{itemize}
    \item 谐振频率的计算
    \item 品质因数Q的定义和计算
    \item 模式的正交性
\end{itemize}

\subsection{圆柱谐振腔}
圆柱形谐振腔的谐振特性。

\section{第十一章:谐振腔链和空间谐波}

\subsection{耦合谐振腔}
多个谐振腔的耦合效应。

\subsection{周期结构}
\begin{itemize}
    \item 布洛赫定理
    \item 色散关系
    \item 带隙结构
\end{itemize}

\section{矢势和运动电荷的场}

\subsection{电磁势}
\begin{itemize}
    \item 标势$\phi$和矢势$\mathbf{A}$的定义
    \item 洛伦兹规范条件:$\nabla \cdot \mathbf{A} + \frac{1}{c^2}\frac{\partial \phi}{\partial t} = 0$
\end{itemize}

\subsection{达朗贝尔方程}
\begin{equation}
    \nabla^2 \phi - \frac{1}{c^2}\frac{\partial^2 \phi}{\partial t^2} = -\frac{\rho}{\varepsilon_0}
\end{equation}

\subsection{李纳-维谢尔势}
运动电荷产生的电磁势。

\section{运动电荷的场和同步辐射}

\subsection{加速电荷的辐射}
\begin{itemize}
    \item 拉莫尔公式
    \item 相对论性推广
\end{itemize}

\subsection{同步辐射}
\begin{itemize}
    \item 同步辐射的特性
    \item 角分布和频谱
    \item 应用领域
\end{itemize}

\section{重要公式汇总}

\subsection{基本常数}
\begin{itemize}
    \item 真空介电常数:$\varepsilon_0 = 8.854 \times 10^{-12}$ F/m
    \item 真空磁导率:$\mu_0 = 4\pi \times 10^{-7}$ H/m
    \item 光速:$c = \frac{1}{\sqrt{\mu_0 \varepsilon_0}} = 3 \times 10^8$ m/s
\end{itemize}

\subsection{常用矢量恒等式}
\begin{itemize}
    \item $\nabla \cdot (\nabla \times \mathbf{A}) = 0$
    \item $\nabla \times (\nabla \phi) = 0$
    \item $\nabla \times (\nabla \times \mathbf{A}) = \nabla(\nabla \cdot \mathbf{A}) - \nabla^2 \mathbf{A}$
\end{itemize}

\section{学习建议和复习要点}

\subsection{重点掌握}
\begin{enumerate}
    \item 麦克斯韦方程组及其物理意义
    \item 电磁波的传播特性和边界条件
    \item 各种导波系统的模式分析
    \item 运动电荷的辐射特性
\end{enumerate}

\subsection{学习方法}
\begin{itemize}
    \item 理论推导与物理图像相结合
    \item 多做习题加深理解
    \item 注意不同坐标系下的数学技巧
    \item 建立知识框架,融会贯通
\end{itemize}

% ===== 参考文献部分 =====
\begin{thebibliography}{99}
\bibitem{jackson} J.D. Jackson, \emph{Classical Electrodynamics}, 3rd ed., Wiley, 1998.
\bibitem{griffiths} D.J. Griffiths, \emph{Introduction to Electrodynamics}, 4th ed., Cambridge University Press, 2017.
\bibitem{guo} 郭硕鸿, \emph{电动力学}, 高等教育出版社, 2008.
\bibitem{born} M. Born and E. Wolf, \emph{Principles of Optics}, 7th ed., Cambridge University Press, 1999.
\end{thebibliography}

\end{document}
